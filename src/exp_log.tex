\subsection{Exponential- und Logartihmusfunktion}
Die Exponentialfunktion ist eine Abbildung $f: \mathbb{R} \rightarrow \mathbb{R}$ mit 
$$f(x) = e^x = \exp{(x)} := \lim_{n\rightarrow \infty}(1+\frac{x}{n})^n , x\in \mathbb{R}$$
Alternativ ist auch folgendes möglich: $e^x := \sum^{\infty}_{n=0} \frac{x^n}{n!}$.\\
Für die Exponentialfunktion gelten alle Potenzgesetze. \\
$f(x) = e^x \Rightarrow f(x) < 0 \quad \forall x \in \mathbb{R}$. $e^x$ ist streng monoton wachsend $\Rightarrow$ $e^x$ ist injektiv. $f: \mathbb{R} \rightarrow (0,\infty)$ ist bijektiv.\\
Die Umkehrfunktion der Exponentialfunktion ist die Logarithmusfunktion. Im Fall von $e^x$ ist $f^-1(x)$ der logarithmus Naturalis (der Logarithmus zur Basis $e$) $\log_e{x} = \ln(x)$. Für den natürlichen Logarithmus gelten alle Logarithmusregeln.